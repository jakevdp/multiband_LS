\documentclass[12pt,preprint]{aastex}

% has to be before amssymb it seems
\usepackage{color,hyperref}
\definecolor{linkcolor}{rgb}{0,0,0.5}
\hypersetup{colorlinks=true,linkcolor=linkcolor,citecolor=linkcolor,
            filecolor=linkcolor,urlcolor=linkcolor}

\usepackage{url}
\usepackage{algorithmic,algorithm}
\usepackage{amssymb,amsmath}

\newcommand{\arxiv}[1]{\href{http://arxiv.org/abs/#1}{arXiv:#1}}

\usepackage{listings}
\definecolor{lbcolor}{rgb}{0.9,0.9,0.9}
\lstset{language=Python,
        basicstyle=\footnotesize\ttfamily,
        showspaces=false,
        showstringspaces=false,
        tabsize=2,
        breaklines=false,
        breakatwhitespace=true,
        identifierstyle=\ttfamily,
        keywordstyle=\bfseries\color[rgb]{0.133,0.545,0.133},
        commentstyle=\color[rgb]{0.133,0.545,0.133},
        stringstyle=\color[rgb]{0.627,0.126,0.941},
    }

\newcommand{\project}[1]{{\sffamily #1}}
\newcommand{\Python}{\project{Python}}
\newcommand{\numpy}{\project{numpy}}
\newcommand{\Ubuntu}{\project{Ubuntu}}
\newcommand{\github}{\project{GitHub}}
\newcommand{\pip}{\project{pip}}
\newcommand{\acor}{\project{acor}}
\newcommand{\thisplain}{emcee}
\newcommand{\this}{\project{\thisplain}}
\newcommand{\paper}{document}
\newcommand{\license}{MIT License}

\newcommand{\foreign}[1]{\emph{#1}}
\newcommand{\etal}{\foreign{et\,al.}}
\newcommand{\etc}{\foreign{etc.}}

\newcommand{\Fig}[1]{Figure~\ref{fig:#1}}
\newcommand{\fig}[1]{\Fig{#1}}
\newcommand{\figlabel}[1]{\label{fig:#1}}
\newcommand{\Tab}[1]{Table~\ref{tab:#1}}
\newcommand{\tab}[1]{\Tab{#1}}
\newcommand{\tablabel}[1]{\label{tab:#1}}
\newcommand{\Eq}[1]{Equation~(\ref{eq:#1})}
\newcommand{\eq}[1]{\Eq{#1}}
\newcommand{\eqlabel}[1]{\label{eq:#1}}
\newcommand{\Sect}[1]{Section~\ref{sect:#1}}
\newcommand{\sect}[1]{\Sect{#1}}
\newcommand{\App}[1]{Appendix~\ref{sect:#1}}
\newcommand{\app}[1]{\App{#1}}
\newcommand{\sectlabel}[1]{\label{sect:#1}}
\newcommand{\Algo}[1]{Algorithm~\ref{algo:#1}}
\newcommand{\algo}[1]{\Algo{#1}}
\newcommand{\algolabel}[1]{\label{algo:#1}}

% math symbols
\newcommand{\dd}{\mathrm{d}}
\newcommand{\like}{\mathscr{L}}
\newcommand{\bvec}[1]{\boldsymbol{#1}}
\newcommand{\paramvector}[1]{\bvec{#1}}
\newcommand{\normal}[2]{\mathcal{N} (#1, #2)}
\newcommand{\ensemble}{S}
\newcommand{\colorens}[1]{\ensemble^{(#1)}}
\newcommand{\red}{\colorens{0}}
\newcommand{\blue}{\colorens{1}}
\renewcommand{\vector}[1]{#1}
\renewcommand{\matrix}[1]{#1}
\newcommand{\pr}[1]{\ensuremath{p(#1)}}
\newcommand{\af}{\ensuremath{a_f}}
\newcommand{\expect}[1]{\left<#1\right>}

% model parameters
\newcommand{\model}{\ensuremath{\vector{\Theta}}}
\newcommand{\data}{\ensuremath{\vector{D}}}
\newcommand{\nuisance}{\ensuremath{\vector{\alpha}}}
\newcommand{\link}{\ensuremath{X}}

% units
\newcommand{\unit}[1]{\mathrm{#1}}

\begin{document}

\title{Multiband Periodograms of Astronomical Sources}

\newcommand{\escience}{1}
\newcommand{\uwastro}{2}
\author{Jacob T. VanderPlas\altaffilmark{\escience}}
\author{{\v Z}eljko Ivezi{\'c}\altaffilmark{\uwastro}}
\altaffiltext{\escience}{eScience Institute, University of Washington}
\altaffiltext{\uwastro}{Department of Astronomy, University of Washington}


\begin{abstract}
  This paper introduces a general method for least-squares spectral fitting of multi-band periodic data.
\end{abstract}

\keywords{
    methods: data analysis ---
    methods: numerical ---
    methods: statistical
}

\section{Introduction}

\begin{itemize}
  \item Lomb Scargle periodograms \& their use in Astronomy
  \item Other methods of period finding (Supersmoother, ARMA-type models, etc.)
  \item Need for multi-band periodograms
\end{itemize}

\section{Multiterm Periodograms: Background}

\begin{itemize}
  \item Relationship between Fourier Periodogram \& Least Squares fit \citep{Jaynes96}
  \item Matrix form of linear models
  \item Generalization of single-term to multi-term models
  \item Show expression for Power; refer to appendix for computational details
\end{itemize}

\section{Moving to Multiple Bands}

\begin{itemize}
  \item Write-out the linear model
  \item Demonstrate the matrix form of this
  \item Use light regularization
  \item Correspondance of the single-term multiband result to the weighted sum of single-band Lomb-Scargle results
\end{itemize}

\section{Examples}
\subsection{Simulated Example}

\begin{itemize}
  \item Show power spectrum for a single-band RR-Lyrae
  \item Reduce observations, show multiple power spectra for multi-band RR-Lyrae
  \item Show single power spectrum for multi-band RR-Lyrae
\end{itemize}

\subsection{Stripe 82}

\begin{itemize}
  \item Compute periods using majority method
  \item Compute periods using unified method
  \item Compare the results
\end{itemize}

\section{Further work}

\begin{itemize}
  \item issues with window function corrections
  \item issues with physicality of model
\end{itemize}


\bibliographystyle{apj}
\bibliography{paper}


\appendix


\section{The Lomb-Scargle Periodogram and Least-Squares Spectral Fitting}

We start with time-series data

\begin{equation}
  D = \{t_i,y_i,\sigma_i\}_{i=1}^N
\end{equation}

without loss of generality, we will assume throughout that the $y_i$ values are centered; that is, the weighted mean

\begin{equation}
  \label{eq:centered-data}
  \frac{\sum_{i=1}^N w_i y_i}{\sum_{i=1}^N w_i} = 0
\end{equation}

where the weights are $w_i \equiv \sigma_i^{-2}$. We will seek a single sinusoid model which fits this data; that is, our model is

\begin{equation}
M(t~|~A,\omega,\phi) = A \sin(\omega t + \phi)
\end{equation}

to make this a linear problem,

\begin{equation}
M(t~|~\omega,\theta) = \theta_0 \sin(\omega t) + \theta_1\cos(\omega t)
\end{equation}

which can be related to the above form by $\theta_0 = A\cos\phi$ and $\theta_1 = A\sin\phi$.
The $\chi^2$ for this model is given by

\begin{equation}
\chi^2(D, \omega, \theta) = \sum_i\frac{[y_i - M(t_i~|~\omega,\theta)]^2}{2\sigma_i^2}
\end{equation}

This can be expressed more compactly by defining the following matrices:

\begin{equation}
X_\omega = \left[
\begin{array}{cc}
\sin(\omega t_1) & \cos(\omega t_1)\\
\sin(\omega t_2) & \cos(\omega t_2)\\
\vdots & \vdots \\
\sin(\omega t_N) & \cos(\omega t_N)\\
\end{array}
\right];~~
y = \left[
\begin{array}{c}
y_1 \\
y_2\\
\vdots \\
y_N\\
\end{array}
\right];~~
\Sigma = \left[
\begin{array}{ccccc}
\sigma_1^2 & 0 & 0 & \cdots & 0\\
0 & \sigma_2^2 & 0 & \cdots & 0\\
0 & 0 & \sigma_3^2 & \cdots & 0\\
\vdots & \vdots & \vdots & \ddots & \vdots\\
0 & 0 & 0 & \cdots & \sigma_N^2
\end{array}
\right]
\end{equation}

Then the model is given by $X_\omega\theta$ and the $\chi^2$ can be written
compactly as:

\begin{equation}
  \chi^2(\omega, \theta) = (y - X_\omega\theta)^T\Sigma^{-1}(y - X_\omega\theta)
\end{equation}

Minimizing this cost funciton with respect to $\theta$ gives the maximum likelihood parameters:

\begin{equation}
\hat{\theta}_\omega = (X_\omega^T\Sigma^{-1}X_\omega)^{-1}X_\omega^T\Sigma^{-1}y
\end{equation}

We'll simplify this slightly by defining the quantities $\tilde{X}_\omega = \Sigma^{-1/2}X_\omega$ and $\tilde{y} = \Sigma^{-1/2}y$, so that the maximum likelihood parameters can be written

\begin{equation}
  \hat{\theta}_\omega = (\tilde{X}_\omega^T\tilde{X}_\omega)^{-1}\tilde{X}_\omega^T\tilde{y}
\end{equation}

Now the value of $\chi^2$ at maximum likelihood can be expressed:

\begin{equation}
  \chi^2(\omega, \hat{\theta}_\omega) = \tilde{y}^T\tilde{y} - \tilde{y}^T\tilde{X}_\omega(\tilde{X}_\omega^T\tilde{X}_\omega)^{-1}\tilde{X}_\omega^T \tilde{y}
\end{equation}

The form of this equation suggests defining a reference $\chi^2$ value which is given by a best-fit constant model to the data: because of the constraint given in Eqn.~\ref{eq:centered-data}, this reference is simply written $\chi^2_0 = \tilde{y}^T\tilde{y}$. We can now follow Lomb [ref] \& Scargle [ref] and write the normalized periodogram:

\begin{eqnarray}
  \label{eq:power}
  P(\omega) &\equiv& 1 -\frac{\chi^2(\omega, \hat{\theta}_\omega)}{\chi^2_0(\omega)}\\ &=& \frac{\tilde{y}^T\tilde{X}_\omega(\tilde{X}_\omega^T\tilde{X}_\omega)^{-1}\tilde{X}_\omega^T \tilde{y}}{\tilde{y}^T\tilde{y}}
\end{eqnarray}

Typically, the Lomb-Scargle periodogram is expressed in terms of various weighted sums of sines and cosines of the data; this expression is equivalent, but has the advantage that it can be trivially generalized to an arbitrary linear model through adding columns to the $X$ matrix, and can handle arbitrary correlated measurement errors by adding appropriate off-diagonal terms to the $\Sigma$ matrix.

\section{Higher-order Models}

As an example of one of these generalizations, consider the {\it generalized Lomb-Scargle} method of \citep{Zechmeister09}. This adjusts the classic Lomb-Scargle algorithm by using a model with a floating mean, which can be more accurate for certain observing cadences:

\begin{equation}
  M(t~|~\omega, \theta) = \theta_0 + \theta_1\sin\omega t + \theta_2\cos\omega t
\end{equation}

The normalized power under this model can be computed by simply adding a column of ones to the $X$ matrix.

In a similar vein, the power for a truncated Fourier series model of any order can be constructed by extending the $X$ matrix with appropriate columns; for example, for a 2-term truncated Fourier fit with a floating mean we can write

\begin{equation}
X_\omega = \left[
\begin{array}{ccccc}
1 & \sin(\omega t_1) & \cos(\omega t_1) & \sin(2\omega t_1) & \cos(2\omega t_1)\\
1 & \sin(\omega t_2) & \cos(\omega t_2) & \sin(2\omega t_2) & \cos(2\omega t_2)\\
1 & \sin(\omega t_3) & \cos(\omega t_3) & \sin(2\omega t_3) & \cos(2\omega t_3)\\
\vdots & \vdots & \vdots & \vdots & \vdots \\
1 & \sin(\omega t_N) & \cos(\omega t_N) & \sin(2\omega t_N) & \cos(2\omega t_N)\\
\end{array}
\right]
\end{equation}

Plugging this $X$ matrix into Eqn.~\ref{eq:power} will give us the normalized periodogram associated with the two-term truncated Fourier model.

\section{Regularized Models}

(Show how regularization works its way through the equations \& derive expression for power).


\end{document}
